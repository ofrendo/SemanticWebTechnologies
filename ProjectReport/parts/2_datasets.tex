\section{Datasets used} 
Which datasets does the application use?
- DBPedia, DBPediaLive, iServe, ... 
How are they accessed (SPARQL, local)?
How do you combine information from different datasets? 

In order to identify as many named entities as possible with one source we started with the DBPedia\footnote{\url{http://wiki.dbpedia.org/}} dataset using the public data endpoint\footnote{\url{http://dbpedia.org/sparql}} and tested with the corresponding Virtuoso\footnote{\url{https://virtuoso.openlinksw.com/}} SPARQL explorer\footnote{\url{http://dbpedia.org/snorql/}}. So the first version was focused on querying DBPedia but the query generation was rewritten such that it can be used in general for all sources supporting SPARQL 1.1 now. The technical details are described in the next chapter. Per source only the endpoint URL as well as the source specific RDF Type URIs for organisations, locations and persons have to be specified, e.g. http://dbpedia.org/ontology/Organisation for DBPedia. This information is needed to incorporate the entity type information retrieved from the named entity recognitions and as filter for entity search. The filter serves two purposes: performance is increased because only labels of enities of this type are considered for matching the name and to increase the chance to find the correct entity, assuming that the retrieved entity type is correct. 

In the most recent version multiple sources are configured additionally to DBPedia, but many are not actively used, as described in table \ref{tab:sources}.
\begin{table}[H]
	\begin{tabular*}{\textwidth}{p{2,2cm}|p{3,8cm}|p{2,1cm} |p{3cm}}
		
		\textbf{Data Set} &\small \textbf{SPARQL Endpoint} & \textbf{Entities} & \textbf{Usage}  \\
		\hline 
		\textbf{DBPediaLive} &\small \url{http://dbpedia-live.openlinksw.com/sparql/} & Organisation, Person, Location  & Active per default\\
		\hline 
		\textbf{iServer} &\small \url{http://iserve.kmi.open.ac.uk/iserve/sparql} & Organisation & Active per default \\
		\hline 
		\textbf{FactForge} &\small \url{http://factforge.net/sparql} & Organisation, Person, Location & Timeout \\
		\hline 
		\textbf{European Environment Agency} &\small \url{http://semantic.eea.europa.eu/sparql} &  Organisation, Person, Location  & Error: only supports SPARQL 1.0  \\
		\hline   
		\textbf{LinkedMDB} &\small \url{http://linkedmdb.org/sparql} &  Organisation, Person, Location  & Error: only supports SPARQL 1.0 \\
		\hline 
		\textbf{Education (UK)} & \small \url{http://services.data.gov.uk/education/sparql} &  Organisation, Location  & Slow, and sameAs definitions are missing \\
		\hline 
		\textbf{DataGov (UK)} &\small \url{http://services.data.gov.uk/reference/sparql} &  Organisation, Person  & Not useful, only internal links\\
		\hline 
		\textbf{World Bank} &\small \url{http://worldbank.270a.info/sparql} &  Location & Error: No rdfs:label, uses skos:prefLabel \\
		\hline 
		\textbf{YAGO2} &\small \url{https://linkeddata1.calcul.u-psud.fr/sparql} &  Organisation, Person, Location & Active per default, but slowest \\
	\end{tabular*}
	\caption{Additional data sources and their usage}
	\label{tab:sources}
\end{table}

Most of the source were identified via \url{http://sparqles.ai.wu.ac.at}. This website monitor hundreds of SPARQL endpoints and allows to filter based on interoperability (SPARQL 1.1) and providing an availability chart. That helped a lot to judge whether or not a endpoint could be usable from a technical point of view. Afterwards we explored the data source by selecting the available RDF types and searching for potentially interesting properties. 