% do not change these two lines (this is a hard requirement
% there is one exception: you might replace oneside by twoside in case you deliver 
% the printed version in the accordant format
\documentclass[11pt,titlepage,oneside,openany]{article}
\usepackage{times}

\usepackage{url}
\usepackage[hidelinks]{hyperref}
\usepackage{graphicx}
\usepackage{latexsym}
\usepackage{amsmath}
\usepackage{amssymb}

\usepackage{ntheorem}

% \usepackage{paralist}
\usepackage{tabularx}

% this packaes are useful for nice algorithms
\usepackage{algorithm}
\usepackage{algorithmic}

% well, when your work is concerned with definitions, proposition and so on, we suggest this
% feel free to add Corrolary, Theorem or whatever you need
\newtheorem{definition}{Definition}
\newtheorem{proposition}{Proposition}


% its always useful to have some shortcuts (some are specific for algorithms
% if you do not like your formating you can change it here (instead of scanning through the whole text)
\renewcommand{\algorithmiccomment}[1]{\ensuremath{\rhd} \textit{#1}}
\def\MYCALL#1#2{{\small\textsc{#1}}(\textup{#2})}
\def\MYSET#1{\scshape{#1}}
\def\MYAND{\textbf{ and }}
\def\MYOR{\textbf{ or }}
\def\MYNOT{\textbf{ not }}
\def\MYTHROW{\textbf{ throw }}
\def\MYBREAK{\textbf{break }}
\def\MYEXCEPT#1{\scshape{#1}}
\def\MYTO{\textbf{ to }}
\def\MYNIL{\textsc{Nil}}
\def\MYUNKNOWN{ unknown }
% simple stuff (not all of this is used in this examples thesis
\def\INT{{\mathcal I}} % interpretation
\def\ONT{{\mathcal O}} % ontology
\def\SEM{{\mathcal S}} % alignment semantic
\def\ALI{{\mathcal A}} % alignment
\def\USE{{\mathcal U}} % set of unsatisfiable entities
\def\CON{{\mathcal C}} % conflict set
\def\DIA{\Delta} % diagnosis
% mups and mips
\def\MUP{{\mathcal M}} % ontology
\def\MIP{{\mathcal M}} % ontology
% distributed and local entities
\newcommand{\cc}[2]{\mathit{#1}\hspace{-1pt} \# \hspace{-1pt} \mathit{#2}}
\newcommand{\cx}[1]{\mathit{#1}}
% complex stuff
\def\MER#1#2#3#4{#1 \cup_{#3}^{#2} #4} % merged ontology
\def\MUPALL#1#2#3#4#5{\textit{MUPS}_{#1}\left(#2, #3, #4, #5\right)} % the set of all mups for some concept
\def\MIPALL#1#2{\textit{MIPS}_{#1}\left(#2\right)} % the set of all mips





\begin{document}

\pagenumbering{roman}
% lets go for the title page, something like this should be okay
\begin{titlepage}
	\vspace*{2cm}
  \begin{center}
   {\Large IE650 Semantic Web Technologies\\}
   \vspace{2cm} 
   {Project Report\\}
   \vspace{2cm}
   {presented by\\
    Oliver Frendo (1510432) \\
    Sascha Ulbrich (1493181) \\
   }
   \vspace{1cm} 
   {submitted to the\\
    Data and Web Science Group\\
    Prof.\ Dr.\ Paulheim\\
    University of Mannheim\\} \vspace{2cm}
   {December 2016}
  \end{center}
\end{titlepage} 

% no lets make some add some table of contents
\tableofcontents
\newpage

%\listofalgorithms

%\listoffigures

%\listoftables

% evntuelly you might add something like this
% \listtheorems{definition}
% \listtheorems{proposition}

\newpage


% okay, start new numbering ... here is where it really starts
\pagenumbering{arabic}

• 10-12 pages (sharp!) without title and toc pages 
• due Friday, December 2nd, 23:59 
• send by email to Anna Lisa, André and Heiko 
• describe your solution including the steps to get there (chapters already created)

Requirements 
You must use the DWS master thesis layout
Please cite sources properly. Preferred citation style [Author, year]



\section{Application domain and goals}
Which users are targeted? – Which user problems are solved? – Which user information needs are addressed
%@Olli?

\section{Datasets used} 
Which datasets does the application use?
- DBPedia, DBPediaLive, iServe, ... 
How are they accessed (SPARQL, local)?
How do you combine information from different datasets? 

In order to identify as many named entities as possible with one source we started with the DBPedia\footnote{\url{http://wiki.dbpedia.org/}} dataset using the public data endpoint\footnote{\url{http://dbpedia.org/sparql}} and tested with the corresponding Virtuoso\footnote{\url{https://virtuoso.openlinksw.com/}} SPARQL explorer\footnote{\url{http://dbpedia.org/snorql/}}. So the first version was focused on querying DBPedia but the query generation was rewritten such that it can be used in general for all sources supporting SPARQL 1.1 now. The technical details are described in the next chapter. Per source only the endpoint URL as well as the source specific RDF Type URIs for organisations, locations and persons have to be specified, e.g. http://dbpedia.org/ontology/Organisation for DBPedia. This information is needed to incorporate the entity type information retrieved from the named entity recognitions and as filter for entity search. The filter serves two purposes: performance is increased because only labels of enities of this type are considered for matching the name and to increase the chance to find the correct entity, assuming that the retrieved entity type is correct. 

In the most recent version multiple sources are configured additionally to DBPedia, but many are not actively used, as described in table \ref{tab:sources}.
\begin{table}[ht]
	\begin{tabular*}{\textwidth}{p{2,2cm}|p{3,8cm}|p{2,1cm} |p{3cm}}
		
		\textbf{Source} &\small \textbf{SPARQL Endpoint} & \textbf{Entities} & \textbf{Usage}  \\
		\hline 
		\textbf{DBPediaLive} &\small \url{http://dbpedia-live.openlinksw.com/sparql/} & Organisation, Person, Location  & Active per default\\
		\hline 
		\textbf{iServer} &\small \url{http://iserve.kmi.open.ac.uk/iserve/sparql} & Organisation & Active per default \\
		\hline 
		\textbf{FactForge} &\small \url{http://factforge.net/sparql} & Organisation, Person, Location & Timeout \\
		\hline 
		\textbf{European Environment Agency} &\small \url{http://semantic.eea.europa.eu/sparql} &  Organisation, Person, Location  & Error: only supports SPARQL 1.0  \\
		\hline   
		\textbf{LinkedMDB} &\small \url{http://linkedmdb.org/sparql} &  Organisation, Person, Location  & Error: only supports SPARQL 1.0 \\
		\hline 
		\textbf{Education (UK)} & \small \url{http://services.data.gov.uk/education/sparql} &  Organisation, Location  & Slow, and sameAs definitions are missing \\
		\hline 
		\textbf{DataGov (UK)} &\small \url{http://services.data.gov.uk/reference/sparql} &  Organisation, Person  & Not useful, only internal links\\
	\end{tabular*}
	\caption{Additional data sources and their usage}
	\label{tab:sources}
\end{table}

Most of the source were identified via \url{http://sparqles.ai.wu.ac.at}. This website monitor hundreds of SPARQL endpoints and allows to filter based on interoperability (SPARQL 1.1) and providing an availability chart. That helped a lot to judge whether or not a endpoint could be usable from a technical point of view. Afterwards we explored the data source by selecting the available RDF types and searching for potentially interesting properties. 



\section{Techniques used} 
Reasoning
Search
external services

\section{Example results}
What outcomes does the application provide? – How is are some user queries answered? 
%@Olli?

\section{Known limitations}
In which domains does the application not work? 
Are there queries which cannot be answered? – Why? 

How could you overcome those limitations, given more time?

We could mention papers describing techniques which could improve certain areas like LINDA \cite{boehm_linda:_2012} for entity matching. 


\section{Lessons learned}
Which challenges did you face?
- Named Entity Identification as URI
- Performance
- Rejected queries
- no sameAs definitions between data sets


What were the biggest obstacles?
- finding active and stable public SPARQL endpoints


What would you do differently next time?
- using local sources
%References
%Add own references in .bib file
\pagebreak
\bibliographystyle{apalike}
\bibliography{SWTReport_Sascha}  
%\bibliography{SWTReport_Sascha,SWTReport_Olli}


\end{document}
