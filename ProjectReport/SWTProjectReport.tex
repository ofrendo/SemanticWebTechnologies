% do not change these two lines (this is a hard requirement
% there is one exception: you might replace oneside by twoside in case you deliver 
% the printed version in the accordant format
\documentclass[11pt,titlepage,oneside,openany]{article}
\usepackage{times}

\usepackage{url}
\usepackage[hidelinks]{hyperref}
\usepackage{graphicx}
\usepackage{latexsym}
\usepackage{amsmath}
\usepackage{amssymb}

\usepackage{ntheorem}

% \usepackage{paralist}
\usepackage{tabularx}

% this packaes are useful for nice algorithms
\usepackage{algorithm}
\usepackage{algorithmic}

% well, when your work is concerned with definitions, proposition and so on, we suggest this
% feel free to add Corrolary, Theorem or whatever you need
\newtheorem{definition}{Definition}
\newtheorem{proposition}{Proposition}


% its always useful to have some shortcuts (some are specific for algorithms
% if you do not like your formating you can change it here (instead of scanning through the whole text)
\renewcommand{\algorithmiccomment}[1]{\ensuremath{\rhd} \textit{#1}}
\def\MYCALL#1#2{{\small\textsc{#1}}(\textup{#2})}
\def\MYSET#1{\scshape{#1}}
\def\MYAND{\textbf{ and }}
\def\MYOR{\textbf{ or }}
\def\MYNOT{\textbf{ not }}
\def\MYTHROW{\textbf{ throw }}
\def\MYBREAK{\textbf{break }}
\def\MYEXCEPT#1{\scshape{#1}}
\def\MYTO{\textbf{ to }}
\def\MYNIL{\textsc{Nil}}
\def\MYUNKNOWN{ unknown }
% simple stuff (not all of this is used in this examples thesis
\def\INT{{\mathcal I}} % interpretation
\def\ONT{{\mathcal O}} % ontology
\def\SEM{{\mathcal S}} % alignment semantic
\def\ALI{{\mathcal A}} % alignment
\def\USE{{\mathcal U}} % set of unsatisfiable entities
\def\CON{{\mathcal C}} % conflict set
\def\DIA{\Delta} % diagnosis
% mups and mips
\def\MUP{{\mathcal M}} % ontology
\def\MIP{{\mathcal M}} % ontology
% distributed and local entities
\newcommand{\cc}[2]{\mathit{#1}\hspace{-1pt} \# \hspace{-1pt} \mathit{#2}}
\newcommand{\cx}[1]{\mathit{#1}}
% complex stuff
\def\MER#1#2#3#4{#1 \cup_{#3}^{#2} #4} % merged ontology
\def\MUPALL#1#2#3#4#5{\textit{MUPS}_{#1}\left(#2, #3, #4, #5\right)} % the set of all mups for some concept
\def\MIPALL#1#2{\textit{MIPS}_{#1}\left(#2\right)} % the set of all mips





\begin{document}

\pagenumbering{roman}
% lets go for the title page, something like this should be okay
\begin{titlepage}
	\vspace*{2cm}
  \begin{center}
   {\Large IE650 Semantic Web Technologies\\}
   \vspace{2cm} 
   {Project Report\\}
   \vspace{2cm}
   {presented by\\
    Oliver Frendo (1510432) \\
    Sascha Ulbrich (1493181) \\
   }
   \vspace{1cm} 
   {submitted to the\\
    Data and Web Science Group\\
    Prof.\ Dr.\ Paulheim\\
    University of Mannheim\\} \vspace{2cm}
   {December 2016}
  \end{center}
\end{titlepage} 

% no lets make some add some table of contents
\tableofcontents
\newpage

%\listofalgorithms

%\listoffigures

%\listoftables

% evntuelly you might add something like this
% \listtheorems{definition}
% \listtheorems{proposition}

\newpage


% okay, start new numbering ... here is where it really starts
\pagenumbering{arabic}

• 10-12 pages (sharp!) without title and toc pages 
• due Friday, December 2nd, 23:59 
• send by email to Anna Lisa, André and Heiko 
• describe your solution including the steps to get there (chapters already created)

Requirements 
You must use the DWS master thesis layout
Please cite sources properly. Preferred citation style [Author, year]



\section{Application domain and goals}
Which users are targeted? – Which user problems are solved? – Which user information needs are addressed

\section{Datasets used} 
Which datasets does the application use? – How are they accessed (SPARQL, local)? – How do you combine information from different datasets? 

\section{Techniques used} 
Reasoning – Search – external services

\section{Example results}
What outcomes does the application provide? – How is are some user queries answered? 


\section{Known limitations}
– In which domains does the application not work? – Are there queries which cannot be answered? – Why? – How could you overcome those limitations, given more time?

We could mention papers describing techniques which could improve certain areas like LINDA \cite{boehm_linda:_2012} for entity matching. 

\section{Lessons learned}
– Which challenges did you face? – What were the biggest obstacles? – What would you do differently next time?

%References
%Add own references in .bib file
\pagebreak
\bibliographystyle{apalike}
\bibliography{SWTReport_Sascha}  
%\bibliography{SWTReport_Sascha,SWTReport_Olli}


\end{document}
